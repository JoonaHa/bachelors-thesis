 \begin{abstract}

Järjestelmäintegraatiot ovat nykyään vakiintunut osa ohjelmistoekosysteemiä. Tarve järjestelmäintegraatioille syntyi organisaatioiden datan saarekkeistua. Saarekkeet eivät pystyneet jakamaan dataa keskenään ja järjestelmien välinen keskustelu oli lähes mahdotonta. Järjestelmäintegraatio ratkaisuja löytyy monenlaisia: kaupallisia ja avoimen lähdekoodin ratkaisuja, isoilta ja pieniltä toimijoilta.  
Tästä huolimatta järjestelmäintegraatio ratkaisujen lähestymistavoista ja suunnittelumalleista löytyy paljon yhtäläisyyksiä.

Tämän tutkielman tarkoituksena on tutustua kirjallisuuteen, joka on vaikuttanut nykypäivän integraatioratkaisujen kehitykseen ja mistä ratkaisujen yhtäläisyydet kumpuavat. Erityisesti huomiota annetaan sanomapohjaisille suunnittelumalleille.
Tutkielmassa perehdytään erilaisiin lähestymistapoihin, suunnittelumallien kategorisointiin ja suunnittelumallien rakennuspaloihin. Sanomapohjaisten suunnittelumallien osalta perehdytään niiden sijoittumiseen nykypäivän teknologiaekosysteemiin, miten niitä on modernisoitu, mitkä ovat sanomapohjaisten suunnittelumallien tunnetut puutteet ja miten nämä suunnittelumallit ovat kehittyneet vuosien saatossa eri implementaatioissa ja tieteellisessä kirjallisuudessa.

 \end{abstract}

%\begin{otherlanguage}{english}
%\begin{abstract}
%Write your abstract here.
%
%In addition, make sure that all the entries in this form are completed.
%
%Finally, specify 1--3 ACM Computing Classification System (CCS) topics, as per \url{https://dl.acm.org/ccs}.
%Each topic is specified with one path, as shown in the example below, and elements of the path separated with an arrow.
%Emphasis of each element individually can be indicated
%by the use of bold face for high importance or italics for intermediate
%level.
%
%\end{abstract}
%\end{otherlanguage}
