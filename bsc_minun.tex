
\chapter{Johdanto}


Järjestelmäintegraatiot tai englanniksi Enterprise Application Integration (EAI) tarkoittaa ohjelmistoja, 
jotka yhdistävät eli integroivat organisaation olemassa olevia erillisiä järjestelmiä toimimaan yhdessä.

Tarve EAI:lle kehittyi kun organisaation jo olemassa olevien
järjestelmien hajautunut tieto piti yhdistää tiedonkulun optimoimiseksi ja
automatisoimiseksi. Esimerkiksi organisaation asiakkuudenhallinnasta vastaavan ohjelmiston
pitäisi pystyä keskustelemaan myynninhallinnointiohjelmiston kanssa pitääkseen
asiakastilastot ajan tasalla. 
Järjestelmät jäivät helposti erillisiksi saarekkeiksi eivätkä jakaneet tietoa keskenään. 
Syntyi niin sanottuja "savupiippujärjestelmiä", joissa järjestelmä oltiin suunniteltu pitämään data esimerkiksi firman yksittäisen osaston sisällä, eikä järjestelmän suunnitteluvaiheessa oltu huomioitu datan jakamista; järjestelmästä puuttui esimerkiksi ulospäin suuntavat ohjelmointirajapinnat. 

Järjestelmäintegraatio mahdollistaa näiden eri järjestelmien keskustelevan toisilleen. Näin voidaan rakentaa yhteinen järjestelmä jo olemassa olevista palasista \citep[sivu 15]{linthicum2000enterprise}. Useissa tapauksissa organisaation sisäisiä järjestelmiä yhdistetään myös ulkoisiin järjestelmiin tai muiden organisaatioiden kanssa \cite{Johannesson2001}.
Organisaation olemassa olevat järjestelmät voi olla kehitetty eri käyttöjärjestelmille, hyödyntävät eri
ohjelmointikieliä tai tietokantaratkaisuita. Väliin tarvitaan integraatio, joka hoitaa
kommunikoinnin ohjelmistojen välillä


Vuonna 2020 EAI-markkinan arvioitiin olevan 9,26 miljardin dollarin (USD) arvoinen \citep{mordorintelligence} ja koostuu ohjelmistojättien (Microsoft, Oracle, Salesforce IBM) ja pienempien yksittäisten yritysten (Workato, SnapLogic, Tibico) tarjoamista järjestelmistä.


\chapter{Integraadioiden lähestymistavat}

\section{Tekniset lähestymistavat}

Järjestelmäintegraatioiden arkkitehtuurin määrää pitkälti liiketoiminnan tarpeet ja niiden asettamat vaatimukset. Integraatioiden tekniset lähestymistavat voi luokitella neljään erilaiseen yläkategoriaan \citep{Hohpe2004}.

\begin{enumerate}
   \item Tiedostojen siirto
   \item Jaettu tietokanta
   \item Etäproseduurikutsu
   \item Viestintä
   
\end{enumerate}



\section{Integraatiotasot}

Teoksessaan \citep{linthicum2000enterprise} kategorioi itegraatiometodit neljään eri tasoon

\begin{enumerate}
   \item Datataso: Dataa liikutellaan eri tietovarastojen välillä ja mahdollinen datan prosessointi ja muokkaus toteutetaan siirron yhteydessä. Kyseisellä menetelmällä on myös paljon yhtäläisyyksiä tyypillisen datavaraston (data warehouse) toteutuksen kanssa.
   \item Rajapintataso: Tasolla pyritään hyödyntämään tarkoitusta varten tilattujen tai paketoitujen sovellusten, esimerkiksi SAP, PeopleSoft, tarjoamia ohjelmointirajapintoja.
   \item Metoditaso: Tällä tasolla pyritään hyödyntämään jaettua sovelluslogiikka. Esimerkiksi hyöntyntämällä sovelluspalvelimia (application server) metodienjakamiseen tai käyttämällä hajautettuja objekteja tai etäkutsuja hyödyntäviä teknologioita.
   \item Käyttöliittymätaso: Integraatioiden yhdistävä tekijä on olemassa olevan sovelluksen käyttöliittymä kun sovellus ei tarjoa muita keinoja datan jakamiseen. Tämä toteutetaan hyödyntymällä ruudun tiedonharavointi tekniikoita (screen scraping). \citep{linthicum2000enterprise}
   
\end{enumerate}


\chapter{Viestintä}

\chapter{Integraatioteknologioiden jaottelusta}

\chapter{Yhteenveto}
