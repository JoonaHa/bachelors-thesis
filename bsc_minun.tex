
\chapter{Johdanto}

\chapter{Järjestelmäintegraatio}


Järjestelmäintegraatiot tai englanniksi Enterprise Application Integration (EAI) tarkoittaa ohjelmistoja, jotka yhdistävät eli integroivat organisaation olemassa olevia erillisiä järjestelmiä toimimaan yhdessä. Tarve EAIlle nousi kun, organisaatioiden automaatisoidessa järjestelmiään, järjestelmät jäivät helposti erillisiksi saarekkeiksi eivätkä jakaneet tietoa keskenään. Syntyi niin sanottuja "savupiippujärjestelmiä", joissa data kulki vain yhdessä piipussa. 

Järjestelmäintegraatio mahdollistaa näiden eri järjestelmien keskustelevan toisilleen vaikka ne käyttäisivät eri käyttöjärjestelmiä, ohjelmointikieliä tai tietokantaratkaisuja. Näin voidaan rakentaa yhteinen järjestelmä jo olemassa olevista palasista \cite{linthicum2000enterprise}. Useissa tapauksissa organisaation sisäisiä järjestelmiä yhdistetään myös ulkoisiin järjestelmiin tai muiden orginisaatioiden kanssa \cite{Johannesson2001}.

Järjestelmäintegraatiot tai englanniksi Enterprise Application Integration (lyh. EAI) tarkoittaa
ohjelmistoja jotka yhdistävät eli integroivat firman tai firmojen olemassa olevia erillisiä
järjestelmiä toimimaan yhdessä. Tarve EAI:lle syntyy kun organisaation jo olemassa olevien
järjestelmien hajautunut tieto pitäisi yhdistää tiedonkulut optimoimiseksi ja
automatisoimiseksi. Esimerkiksi organisaation asiakkuudenhallinnasta vastaavan ohjelmiston
pitäisi pystyä keskustelemaan myynninhallinnointiohjelmiston kanssa pitääkseen
asiakastilastot ajan tasalla; ohjelmistot voivat pyöriä eri käyttöjärjestelmällä ja käyttää
ohjelmointikieliä ja tietokantaratkaisuita. Väliin tarvitaan integraatio, joka hoitaa
kommunikoinnin ohjelmistojen välillä.

Vuonna 2020 EAI-markkinan arvioitiin olevan 9,26 miljardin dollarin (USD) arvoinen \citep{mordorintelligence} ja koostuu ohjelmistojättien (Microsoft, Oracle, Salesforce IBM) ja pienempien yksittäisten yritysten (Workato, SnapLogic, Tibico) tarjoamista järjestelmistä.


\section{Järjestelmäintegraatioiden määritelmä}
Kirjassa Enterprise Integration Patterns (lyh. EIP) Gregor Hohpe ja Bobby Woolf \citep{10.5555/940308}
erottelevat integraatiot hajautetuista sovelluksista määrittelemällä, että vaikka
yritysohjelmistot usein hyödyntävät monitasoista arkkitehtuuria (tai n-tier arkkitehtuuri) jossa
eri koneilla olevat prosessit kommunikoivat keskenään, on tämä sovelluksen hajauttamista
eikä integraatiota. Ensisijaisesti, monitasoisessa arkkitehtuurissa keskenään kommunikoivat
komponentit ovat tiukasti riippuvaisia toisistaan kuin taas integraatioissa komponentit ovat
itsenäisiä sovelluksia, jotka ovat löyhästi kytköksissä eivätkä ole riippuvaisia toistaan
toimiakseen itsenäisesti. Toiseksi, monitasoisen arkkitehtuurin tasot kommunikoivat
useimmiten synkronisesti, kun taas integraatiot hyödyntävät asynkronista kommunikointia ja
voivat edetä suorituksesta ilman vastausta tai vaihtaa rinnakkaiseen tehtävään kunnes vastaus
on saapunut. Kolmanneksi monitasoisilla sovelluksilla on usein ihminen käyttäjänä, joka
odottaa suht’ nopeaa vastausta, kun taas integroiduilla applikaatioilla on vapaampi aikarajoite,
koska ohjelmisto voi suorittaa toista tehtävää odottaessaan vastausta, eikä omista
ihmiskäyttäjän kärsimättömyyttä.


\chapter{Yhteenveto}
