
\chapter{Johdanto}

\chapter{Järjestelmäintegraatio}


Järjestelmäintegraatiot tai englanniksi Enterprise Application Integration (EAI) tarkoittaa ohjelmistoja, jotka yhdistävät eli integroivat organisaation olemassa olevia erillisiä järjestelmiä toimimaan yhdessä. Tarve EAIlle nousi kun, organisaatioiden automaatisoidessa järjestelmiään, järjestelmät jäivät helposti erillisiksi saarekkeiksi eivätkä jakaneet tietoa keskenään. Syntyi niin sanottuja "savupiippujärjestelmiä", joissa data kulki vain yhdessä piipussa. 

Järjestelmäintegraatio mahdollistaa näiden eri järjestelmien keskustelevan toisilleen vaikka ne käyttäisivät eri käyttöjärjestelmiä, ohjelmointikieliä tai tietokantaratkaisuja. Näin voidaan rakentaa yhteinen järjestelmä jo olemassa olevista palasista \cite{linthicum2000enterprise}. Useissa tapauksissa organisaation sisäisiä järjestelmiä yhdistetään myös ulkoisiin järjestelmiin tai muiden orginisaatioiden kanssa \cite{Johannesson2001}.


Vuonna 2020 EAI-markkinan arvioitiin olevan 9,26 miljardin dollarin (USD) arvoinen \citep{mordorintelligence} ja koostuu ohjelmistojättien (Microsoft, Oracle, Salesforce IBM) ja pienempien yksittäisten yritysten (Workato, SnapLogic, Tibico) tarjoamista järjestelmistä.

\section{Järjestelmäintegraatioiden määritelmä}



\chapter{Yhteenveto}