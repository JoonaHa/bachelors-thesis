
\chapter{Johdanto}


Järjestelmäintegraatiot tai englanniksi Enterprise Application Integration (EAI) tarkoittaa ohjelmistoja, 
jotka yhdistävät eli integroivat organisaation olemassa olevia erillisiä järjestelmiä toimimaan yhdessä.

Tarve EAI:lle kehittyi kun organisaation jo olemassa olevien
järjestelmien hajautunut tieto piti yhdistää tiedonkulun optimoimiseksi ja
automatisoimiseksi. Esimerkiksi organisaation asiakkuudenhallinnasta vastaavan ohjelmiston
pitäisi pystyä keskustelemaan myynninhallinnointiohjelmiston kanssa pitääkseen
asiakastilastot ajan tasalla. 
Järjestelmät jäivät helposti erillisiksi saarekkeiksi eivätkä jakaneet tietoa keskenään. 
Syntyi niin sanottuja "savupiippujärjestelmiä", joissa järjestelmä oltiin suunniteltu pitämään data esimerkiksi firman yksittäisen osaston sisällä, eikä järjestelmän suunnitteluvaiheessa oltu huomioitu datan jakamista; järjestelmästä puuttui esimerkiksi ulospäin suuntavat ohjelmointirajapinnat. 

Järjestelmäintegraatio mahdollistaa näiden eri järjestelmien keskustelevan toisilleen. Näin voidaan rakentaa yhteinen järjestelmä jo olemassa olevista palasista \citep[sivu 15]{linthicum2000enterprise}. Useissa tapauksissa organisaation sisäisiä järjestelmiä yhdistetään myös ulkoisiin järjestelmiin tai muiden organisaatioiden kanssa \citep{Johannesson2001}.
Organisaation olemassa olevat järjestelmät voi olla kehitetty eri käyttöjärjestelmille, hyödyntävät eri
ohjelmointikieliä tai tietokantaratkaisuita. Väliin tarvitaan integraatio, joka hoitaa
kommunikoinnin ohjelmistojen välillä


Vuonna 2020 EAI-markkinan arvioitiin olevan 9,26 miljardin dollarin (USD) arvoinen \citep{mordorintelligence} ja koostuu ohjelmistojättien (Microsoft, Oracle, Salesforce IBM) ja pienempien yksittäisten yritysten (Workato, SnapLogic, Tibico) tarjoamista järjestelmistä.


\chapter{Integraadioiden lähestymistavat}

Esittelen tässä luvussa suosituimpia jäsentely ja lähestymistapoja järjestelmäintegraatioille. Eri lähestymistavat perustuvat viitatuimpaan järjestelmäintegraatioarkkitehtuurin kirjallisuuteen. Luvun tarkoituksena on antaa kattokategorioita erilaisille teknisille lähestymistavoille ja antaa historiallista kontekstia eri arkkitehtuurisuuntauksille.

\section{Integraatiotasot}

Teoksessaan \citep{linthicum2000enterprise} luokittelee integraatioiden lähestymistavat neljään eri tasoon

\begin{enumerate}
   \item Datataso: Dataa liikutellaan eri tietovarastojen välillä ja mahdollinen datan prosessointi ja muokkaus toteutetaan siirron yhteydessä. Kyseisellä menetelmällä on myös paljon yhtäläisyyksiä tyypillisen datavaraston \textit{(data warehouse)} toteutuksen kanssa.
   \item Rajapintataso: Tasolla pyritään hyödyntämään tarkoitusta varten tilattujen tai paketoitujen sovellusten, esimerkiksi SAP, PeopleSoft, tarjoamia ohjelmointirajapintoja.
   \item Metoditaso: Tällä tasolla pyritään hyödyntämään jaettua sovelluslogiikka. Esimerkiksi hyöntyntämällä sovelluspalvelimia \textit{(application server)} metodienjakamiseen tai käyttämällä hajautettuja objekteja ja/tai etäkutsuja \textit{(Remote Procedure Call, RPC)} hyödyntäviä teknologioita.
   \item Käyttöliittymätaso: Integraatioiden yhdistävä tekijä on olemassa olevan sovelluksen käyttöliittymä kun sovellus ei tarjoa muita keinoja datan jakamiseen. Tämä toteutetaan hyödyntymällä ruudun tiedonharavointi tekniikoita (screen scraping).
   
\end{enumerate}

\section{Tekniset lähestymistavat}

Järjestelmäintegraatioiden arkkitehtuurin määrää pitkälti liiketoiminnan tarpeet ja niiden asettamat vaatimukset. Integraatioiden tekniset lähestymistavat voi luokitella neljään erilaiseen yläkategoriaan \citep[sivu~64]{Hohpe2004}.

\begin{enumerate}
   \item Tiedostojen siirto
   \item Jaettu tietokanta
   \item Etäproseduuriherätys
   \item Sanomat
   
\end{enumerate}

\textbf{Tiedostojen siirto}: Tiedostojen siirrossa integroitavat sovellukset voivat toimia itsenäisesti ja yhden sovelluksen muutokset eivät vaikuta toisen sovelluksen toimintaan kunhan sovellukset toimivat sovituilla tiedostotyypeillä ja tiedoston nimeämisstrategioilla. Integroijan vastuulla on taata, että tiedostot muutetaan toisen sovelluksen ymmärtämään muotoon. Lähestymistapana tiedostojen siirto on yksinkertainen eikä vaadi lisätyökaluja, koska yleisten tiedostotyyppien (JSON, XML, CSV) tuki löytyy käytännössä jokaisesta ohjelmointikielestä tai integraatiotyökalusta.
Data-analytiikassa ja datavarastojen saralla on myös muita yleisiä tiedostomuotoja kuten Apache Parquet, jolle löytyy myös tuki useista ohjelmointikielistä ohjelmistokirjaston muodossa \citep{Parquet}.
Sovitusta tiedostomuodosta tulee käytännössä integroitavien sovellusten välinen rajapinta.

Tiedostojen siirron heikkouksina on tiedostojärjestelmäoperaaatioiden hallinnointi ja datan synkronointi.
Integraatiokehittäjien vastuulle jää siis tiedostojen lukitseminen kirjoitusoperaatioiden ajaksi tai kirjoitusten ajoittaminen jotta ne eivät mene päällekkäin lukemisen kanssa, tiedostojen nimeämiskäytännöt, tiedostojen arkistointi ja poistaminen. Jos integroitavilla sovelluksilla ei ole pääsyä samoille levyille niin kehittäjien ratkaistavaksi jää myös tiedoston siirtäminen oikealle laitteelle.
Datan synkronointi järjestelmien välillä tuo oman haasteensa, koska tiedostojen siirtoa tapahtuu yleensä harvakseltaan. Esimerkiksi jos asiakkuudenhallintajärjestelmä tuottaa tiedostoja datan synkronointia varten vain kerran päivässä ja laskutusjärjestelmä lähettää laskut aikaisemmin samana päivänä, niin osa laskuista on jo saattanut lähteä vanhaan osoitteeseen, jos asiakas on päivittänyt osoitetietojaan aikaisemmin saman päivän aikana.


\textbf{Jaettu tietokanta}: Jaetun tietokannan etuna tiedostojen siirtoon on muutosten nopea propagoituminen eri sovelluksille. Samassa tietokannassa uusin tieto on saatavilla eri sovelluksille lähes välittömästi. Nopea tiedon liikkuminen tekee virheiden havaitsemisesta ja korjaamisesta helpompaa. Etuna on myös, että tietokantojen datamallit takaavat yhtenevän datan esitysmuodon verrattua tiedostojen siirtoon.
Datan synkronoinnista ja kirjoitus- ja lukuvuoroista vastaa tietokannan hallintajärjestelmä \textit{(DBMS, Database Management System)} ja transaktioiden hallinnointijärjestelmä antaa hyvät työkalut datan eheyden takaamiselle.

Gregor Hohpen ja Bobby Woolfin teos \citep[sivu~69]{Hohpe2004} korostaa myös SQL-pohjaisten relaatiotietokantojen yleisyyttä. Integraatiokehittäjien ei tarvitse opetella uutta teknologiaa tai taistella uuden tiedostoformaatin kanssa vaan kehittäjät voivat työskennellä laajalti tunnettujen relaatiotietokantojen parissa. Valtaosa ohjelmointikielistä ja kehitystyökaluista tukee SQL:n kanssa työskentelyä joten jaetun tietokannan kanssa työskentely on suoraviivaista ja adoptointi helppoa.

Saman tietokannan käyttö estää datan tulkitsemiseen liittyvien ongelmien, kuten semanttisen dissonanssin \textit{(semantic dissonance)} pitkittymistä, missä samaa dataa voidaan tulkita ristiriitaisilla tavoilla. Koska integroitavat sovellukset käyttävät samaa datalähdettä, niin nämä tulkintakysymykset on kohdattava varhaisessa vaiheessa integraatiokehitystä, eikä vasta tuotannossa jossa data voi olla jo yhteensopimatonta tulkintaeroista johtuen. 


Jaetun tietokannan suunnitteluhaasteisiin sisältyy yhtenäisen skeemaan suunnittelu, jota useampi eri sovellus pystyy tehokkaasti hyödyntämään. Usein useamman sovelluksen tuomat vaatimukset johtavat monimutkaiseen tietokantaskeemaan jonka käytön kehittäjät kokevat haastavaksi. Yhtenäisen skeeman suunnittelua voi myös vaikeuttaa "poliittiset" haasteet, koska tietokannan suunnittelu voi johtaa aikataulujen venymiseen ja kommunikaatiohaasteisiin tietokantaa hyödyntävien eri yksiköiden välillä.
Lisää suunnitteluhaasteita tuo ulkoiset ohjelmistot. Lähes poikkeuksetta kaupalliset ohjelmistot tukevat vaan omaa ohjelmiston mukana tulevaa tietokantaformaattia eivätkä taivu siitä poikkeavaan tietokantaskeemaan. Vastaavia haasteita tuo sovellukset jotka on peritty toiselta organisaatiolta esimerkiksi yrityskaupan yhteydessä. Sovellusten jälkeenpäin tehtävä jatkokehittäminen jaettua tietokantaa hyödyntäväksi on yleensä työlästä ja kallista.
Kun jaettuun tietokantaan yhdistettyjen sovellusten määrä lisääntyy niin ratkaisu voi aiheuttaa suorituskykyhaasteita, varsinkin jos luku- ja kirjoitusoperaatiot kohdistuvat vaan muutamaan tietokantatauluun. Jos sovellukset on hajautettu useammalle laitteelle ja tietokanta niiden kanssa, jotta sovelluksilla on lokaali pääsy kantaan, niin hajauttaminen tuo omat haasteensa. Pääosin datan hajauttamistaktiikkojen muodossa ja lisää näin ratkaisun kompleksisuutta.

\textbf{Etäproseduuriherätys:}\textit{(Remote Procedure Invocation)} Edelliset lähestymistavat keskittyivät pääosin datan jakamiseen, mutta näissä lähestymistavoissa pienet datanmuutokset voivat johtaa eri toimintoihin useiden sovellusten taholta. Osoitteen vaihto voi olla yksinkertainen kentän muutos tai laukaista useita rekisteröinti- ja lakiprosesseja useassa eri sovelluksessa. Jaettu tietokanta ei mahdollista minkäänlaista datan kapselointia ja tämä yksi iso datalähde tekee datamuutosten havaitsemisesta ja muutosten vaatimien prosessien aktivoimisesta haastavaa. Tiedostojen siirto tarjoaa suoraviivaisen tavan reagoida datan muutoksen, mutta tämä tapahtuu yleensä viiveellä johtuen tiedostojen synkronoimisen haasteista.
Jaetun tietokannan kapseloimattomuus tarkoittaa myös integraatioiden ylläpidon joustamattomuutta. Muutokset yhdessäkään integroidussa sovelluksessa vaikuttavat jaettuun tietokantaan ja tietokantamuutokset voivat aiheuttaa kauas kantautuvia muutoksia tietokantaa käyttävien sovellusten kesken.

Etäproseduuriherätys mahdollistaa mekanismin jossa sovellus voi kutsua toisen sovelluksen funktiota, jakaa vain tarvittavan datan ja kutsua funktiota joka kertoo datan vastanottajalle miten toimia jaetun datan kanssa.
Jos sovellus tarvitsee toisen sovelluksen dataa se voi kysyä sitä siltä suoraan. Vastaavasti jos sovelluksen tarvitsee muokata toisen sovelluksen dataa niin se voi tehdä funktiokutsun.
Jokainen sovellus vastaa oman datansa eheydestä ja jokainen sovellus voi tehdä muutoksia omaan dataansa, vaikuttamatta muiden sovellusten tilaan.

Etäproseduuriherätyksen mahdollistavat teknologiat ovat myös yleisiä ja tuttuja kehittäjille. Etäproseduurikutsu \textit{(Remote Procedure Call, RPC)} teknologiat ja kirjastot ovat tunnettuja ja yleisesti käytettyjä. Teoksessa \citep[sivu~71]{Hohpe2004} Martin Fowler listaa CORBA, COM, .NET Remoting ja Java RMI esimerkkeinä ja mainitsee, että web-palveluiden yleistyessä http-yhteyksiä hyödyntävät lähestymistavat kuten SOAP ja XML ovat tulleet kehittäjien suosikeiksi. Varsinkin kun http-yhteyksien kanssa on helppo työskennellä, koska useimpien yritysten palomuurit sallivat http-liikenteen. Teoksen julkaisun jälkeen REST ja JSON ovat pitkälti korvanneet SOAP:in ja XML:än web-palveluiden suosituimpana lähestymistapana.

Etäproseduuriherätyksellä on mahdollisuus vähentää  semanttista dissonanssia, koska sovellukset voivat tarjota usean erillaisen rajapinnan samalle datalle. Eri asiakasohjelmille voidaan tarjota erilainen datanesitysmalli riippune siitä mikä asiakasohjelma on kyseessä. Tämä antaa ennemmn mahdollisuuksia esittää datan useammalla eri tavalla verrattuna pelkkään relaationaaliseen malliin.
Useammat eri rajapinnat tarkoittavat lisää työtä integraatiokehittäjille datan muokkaamisen parissa ja integroitaviensovellusten täytyykin neuvotella mitä rajapintoja ne tulevat toisiltansa käyttämään.

Etäproseduuriherätyksen helppous kehittjille voi myös olla sen haittapuoli jos integraatiokehittäjät eivät tiedosta etäkutsujen suorituskyky- ja luotettavuuseroja verrattuna paikkalisiin kutsuihin. Useat etäkutsut voivat kasaannuttaa näitä ongelmia ja johtaa hitaaseen ja epäluotettavaan järjestelmään.

Vaikka etäproseduuriherätyksen mahdollistama datan kapselointi vähentää sovelluksen kytköksiä karsimalla suuren yhteisen datalähteen, niin se voi silti aiheuttaa solmukohtia, erityisesti kun kyse on jaksossa - tietyssä järjestyksessä tehtävistä- operaatioista. Integraatioijärjestelmistä näistä muodostuu helposti ongelma, koska vastaavat kytkökset eivät välttämättä aiheuttaisi ongelmia yksittäisessä sovelluksessa, mutta usemman sovelluksen integraatiossa lisäkytkökset tarkoittavat lisäviivettä ja ylimääräisiä verkkokutsuja.

\textbf{Sanomat:}\textit{(Messaging)} Aikaisemmat integraatioiden lähestymistavat keskittyivät joko datan tai toiminnalisuuden jakamiseen. Gregor Hohpen ja Bobby Woolfin mukaan yleinen integraationkehitykesn haaste on saada eri järjestelmät toimimaan yhdessä mahdollisimman viipettä ilman, että järjestelmien välillä on kytkösiä jotka tekevät järjestelmänstä epäluotettavan joka sovelluksen suorittamisen tai kehittämisen kannalta \citep[sivu~72]{Hohpe2004}. Tiedostojen siirrossa datan siirtyminen ei ole tarpeeksi viipeetöntä ja sovellusten välinen toiminta tarpeeksi sujuvaa vaikka lähestymistapa estääkin rajoittavien kyskösten muodostamisen. Jaetussa tietokannassa data on jaettu ja datan muutokset ovat responsiivisia, mutta kaikki sovellukset ovat kytköksissä samaan tietokantaan ja lähestymistapa ei mahdollista sovellusten yhteistä toimintaa.
Etäproseduuriherätyksen heikkoudet olivat yleiset hajautettujenjärjestelmien sudenkuopat, varsinkin jos eäkutsuja käytetään samallailla kuin paikallisia kutsuja ja lähestymistavassa sovellusten pitää jakaa tietoa toistensa rajapinnoista mikä lisää kehitystä vaikeuttavien kytkösten määrää.

Teoksen mukan \citep[sivu~73]{Hohpe2004} sanomien käyttö erityisesti asynkronisella viestinnällä tarjoaa lähestymistavan mikä on aikaisemmin esiteltyjen lähestymistapojen parhaiden puolien yhdistelmä.
Sanomien käyttö mahdollistaa pienien tiheästi kulkevien datapakettien lähetyksen ja jossa tallenttamisen ja tiedosto-operaatioiden yksityiskohdat on abstsraktoitu pois, joka mahdollistaa nopeille skeeman muutoksille vastaten yrityksen tarpeisiin.
Sovellukset pystyvät jakaa toiminnallisuuksian lähettämällä sanoman toisilleen joka herättää esimerkiksi datan muokkaus proseduurin. Asynkroninen viestintä ei taas vaadi vastaanottajan olemaan saatavilla lähetyhetkellä ja asynkronisen viestintä ohjaa kehittäjiä ymmärtämään, että etäyhteyksien käyttäminen hitaampaa ja suunnittelemaan korkeamman koheesion komponentteja, jolloin etänä tehtävien operaatioiden käyttö on harkitumpaa.
Sanomapohjaiset järjestelmät myös mahdollistavat tiedostojen siirron kaltaisten löyhien kytkösten käytön. Sanomia voidaan muokata kesken lähetyksen ilman, että lähettäjä- tai vastaanottajasovelluksen tarvitsee olla tietoinen muokkausoperaation yksityiskohdista. Tämä mahdollistaa integroijien yleislähettävän \textit{broadcast} sanomia useammalle vastaanottajalle, valitsevan yhden vastaanottajan useamman joukosta tai valita useasta muunlaisesta topologiasta jotka sallivat integraation irrottamisen sovelluksen kehitysprosessista. 
Sanomien tiheä lähettäminen mahdollistaa säännöllisen datan jakamisen lisäksi toiminnallisuutta. Käsittelyprosessi voidaan käynistää heti kun yksittäinen sovellus saapuu ja asynkronisten kutsujen avulla lähettävän sovelluksen suoritus ei keskeydy odottamaan vastausta.


Sanomien lähetyksen tiheä datanvaihto ei kuitenkaan estä semanttisen dissonanssin syntymistä, varsinkin kuin datan esitysmuoto voi vaihtua useasti sanomia muokatessa.
Sanomien lähetyksen tiheys ei myöskään täysin poista samoja datan synkronointi haasteita mitä tiedostojen siirrossa ilmeni. Sanomien siirrossta on jonkin verran viiveitä ja niiden ajoituksella tulee edelleen olemaan merkitystä.
Asynkronisuus tuo myös lisä haasteensa integraatioden kehitys vaiheessa. Testaus ja sovellusten virheenpaikkannus tulee olemaan monimutkaisempaa sanomien lähetyksen rinnakkaisuuden takia ja vaati integraatiokehittäjiltä jonkin verran lisätotuttelua.
Sanomien käytön löyhät kytkennät lisäävät integroitavien sovellusten koheesioita, mutta tarkoittavat kuiten vaikeammin ylläpidettävän "liima koodin" tarvetta jotta integraatiot saadaan toimimaan yhdessä.

Edellä mainitut haasteet tarkoitavat sanomapohjaisille järjestelmille suunniteltuja lähestymistapoja ja arkkitehtuureja mitkä toistuvat järjestelmissä niiden yksittäisistä eroista huolimatta.







\chapter{Sanomat ja sanomavälittäjät}

\chapter{Integraatioteknologioiden jaottelusta}

\chapter{Yhteenveto}
